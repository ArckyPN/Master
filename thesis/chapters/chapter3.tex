\chapter{Requirements\label{cha:chapter3}}

This chapter will outline the components required to fulfill the task of implementing a testbed capable of producing, signing, publishing, playing, and validating a live stream as trustworthy using C2PA Manifests.

\section{Overview\label{sec:reqoverview}}

To reiterate the task, the testbed is supposed to be able to create a live stream. The created live stream is to be signed with a C2PA Manifest and then made available to be watched and also verified that the contained C2PA Manifest is valid. 

\section{Technical Requirements\label{sec:techreq}}

These requirements can be separated into four components:

\begin{enumerate}
    \item \textbf{Producer}: creating a Live Stream
    \item \textbf{Signer}: embedding the C2PA Manifest into the Live Stream
    \item \textbf{Distributor}: hosting the Live Stream for consumption
    \item \textbf{Consumer}: playing and validating the Live Stream
\end{enumerate}

\subsection{Producer}

The Producer must be able to create a live stream that is compliant with the DASH and HLS protocols, ideally simultaneously. In addition, it would be beneficial to be able to create live streams from various inputs, i.e. a camera, screen recording, video file, etc.

Furthermore, the Producer needs to be able to interface with the Signer to be able to sign the Live Stream as it is being created with as little latency impact as possible.

\subsection{Signer}

The Signer has to be able to receive the live stream from the Producer. Once the live stream has been received, it needs to be periodically signed with a C2PA Manifest. Additionally, the signing of the live stream must be optimized to the point where it affects the viewing experience as little as possible.

There already is an open-source implementation of the C2PA specifications: \texttt{c2pa-rs} \footnote{\texttt{c2pa-rs} GitHub Repository: \url{https://github.com/contentauth/c2pa-rs}} written in Rust. This implementation includes the ability to sign fragmented BMFF content and a CLI (command line interface) program. These two offer an ideal baseline for this task.

Once the live stream has been signed, it needs to be published to a Distributor.

\subsection{Distributor}

The Distributor should emulate a CDN. It needs to be able to have content ingested and then digested to viewers in fashion, which does not introduce noticeable delays into the process.

\subsection{Consumer}

Finally, the Consumer must be able to playback the live stream from the CDN using DASH and HLS conform media players. A website is the ideal candidate for this. It offers a robust environment to create a graphical user interface, it is very portable, and is a very common platform to consume media on.

In addition, there are existing implementations for the media players, like \texttt{dash.js} \footnote{\texttt{dash.js} GitHub Repository: \url{https://github.com/Dash-Industry-Forum/dash.js}} and \texttt{hls.js} \footnote{\texttt{hls.js} GitHub Repository: \url{https://github.com/video-dev/hls.js}} for the DASH and HLS protocols, respectively. Likewise there is an existing implementation for validating C2PA Manifests for a wide range of content types: \texttt{c2pa} \footnote{\texttt{c2pa} NPM package: \url{https://www.npmjs.com/package/c2pa}}.