\chapter{Evaluation\label{cha:chapter6}}

In this chapter the implementation of Component X is evaluated. An example instance was created for every service. The following chapter validates the component implemented in the previous chapter against the requirements.
\\
\\
Put some screenshots in this section! Map the requirements with your proposed solution. Compare it with related work. Why is your solution better than a concurrent approach from another organization?


\todo[inline]{figure out what to do with this}
\subsection{Minimizing Re-transmissions of duplicate Data}

The first reason behind the optimization described in the previous section was of cause the time reduction of limiting the number of fragments that need to be signed to a fixed value. However, the second reason is the reduction of transmission overhead. Whenever a fragment is signed, that has already been previously signed, that fragment has to be published anew to the CDN. This is highly problematic because the fragment itself has not actually changed, only the C2PA manifest has changed, so the majority of the transmission is redundant. By only signing a small part of the live stream that number of redundant transmissions is already cut down by a large portion.

This aspect is also the driving thought behind the alternative approaches implemented in this testbed.

By decoupling the C2PA manifests from the fragment entirely, it is possible to simply forward the fragments without alterations to the CDN and allow the CDN to perform its standard procedure caching and at the same time completely eliminate the redundant transmissions of the media data.

\section{Test Environment\label{sec:testenvir}}

Fraunhofer Institute FOKUS' Open IMS Playground was used as a test environment for the telecommunication services. The IMS Playground ...

\section{Scalability\label{sec:scal}}

Lorem Ipsum

\section{Usability\label{sec:usab}}

Lorem Ipsum

\section{Performance Measurements\label{sec:performance}}

Lorem Ipsum