\thispagestyle{empty}
\vspace*{1.0cm}

\begin{center}
    \textbf{Abstract}
\end{center}

\vspace*{0.5cm}

\noindent

It is getting more and more difficult to trust the media you see on the internet. The tools for editing, manipulating, creating and generating media are getting easier to use and more accessible every day.

There are tools, like image filters on social media, which essentially require no expertise to use at all, but also highly specialized tools for manipulating all kinds of media, like images, videos, audio, documents and more, with results that make it harder than ever to distinguish real from forged.

Additionally, there are also now a plethora of tools that are able to generated these kinds of media from just a single text prompt. The results of which are getting more and more convincing as these tools are being iterated upon.

All of this is creating new problems in society, like false perceptions of beauty, unbelievable documentations, faking personalities and content to scam people and many more.

This has lead to the founding of the Coalition of Content Provenance and Authenticity (C2PA) by large companies in the industry. Their goal is to create a standard of embedding metadata into all kinds of data types to document the road media took from creation to the consumption and spread its use to as many places as possible. This metadata is to detail everything involved in the creation, editing, distribution and anything else relevant of the corresponding media.

This data should provide users the option to inform themselves whether the media they are viewing is from a trustworthy source or perhaps may have been tampered with or is from a nefarious source.
