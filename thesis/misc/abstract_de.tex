\thispagestyle{empty}
\vspace*{0.2cm}

\begin{center}
    \textbf{Zusammenfassung}
\end{center}

\vspace*{0.2cm}

\noindent 

Es wird immer schwieriger Media im Internet zu vertrauen. Die Programme zum Bearbeiten, Manipulieren, Erstellen und Generierung von Media werden jeden Tag leichter zu bedienen.

Es gibt Programme, wie zum Bespiel Bildfilter in Soziel Medien, welche praktisch kein Vorwissen benötigen, aber auch hochspezialisierte Programme zur Bearbeitung von jeglichen Arten von Media, wie Bilder, Videos, Audio, Dokumente und mehr, mit Ergebnissen, welche schwerer denn je sind, echten Inhalt von Gefälschten zu unterscheiden.

Dazu kommen zahlreiche Programme, welche in der Lage sind diese Arten von Medien aus nur einer einfachen Textaufforderung zu erzeugen. Deren Ergebnisse mit der Zeit immer überzeugender werden, da sich diese Programma stetig verbessern.

All das führt zu neuen Problem in der Gesellschaft, wie zum Bespiel eine falsche Vorstellung von Schönheit, unglaubwürdigen Dokumenten, Identitätsraub und gefälschten Inhalten, welche verwendet werden, um Leute zu täuschen und auszurauben und Vieles mehr.

Dies hat zu der Gründung der Coalition of Content Provenance and Authenticity (C2PA), das Bündnis für Inhaltsherkunft und -authentizität, durch großen Unternehmen aus der Industrie geführt. Dessen Ziel ist es einen Standart zu entwicklen, welcher Eckdaten in all möglichen Datentypen einzubetten. Diese Daten sollen alle Schritten dokumentieren, die dieser Datentyp von Erstellung bis hin zum Endnutzer durchlaufen ist. Dies schließt jedes Detail ein, welche bei der Erstellung, Bearbeitung, Verteilung und jegliche weitere Information beigetragen haben.

Diesen Daten ermöglichen den Endnutzern die Option sich selbst für den Entstehungsweg des entsprechenden Mediums zu informieren und sicher stellen, dass Dies aus einer vertrauenwürdigen Quelle kommt oder manipuliert wurde oder möglicherweise eine schädlichen Ursprung hat.